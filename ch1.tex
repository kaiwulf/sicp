\documentclass[a4paper,12pt]{article}
\usepackage{amsmath}
\usepackage{amssymb}
% \usepackage{inconsolata}
\begin{document}

1)
\\a) \texttt{10} \rightarrow \texttt{Value: 10}
\\b) \texttt{(+ 5 3 4)} \rightarrow \texttt{Value: 12}
\\c) \texttt{(- 9 1)} \rightarrow \texttt{Value: 8}
\\d) \texttt{(/ 6 2)} \rightarrow \texttt{Value: 3}
\\e) \texttt{(+ (* 2 4) (- 4 6))} \rightarrow \texttt{Value: 6}
\\f) \texttt{(define a 3)} \rightarrow \texttt{Value: 3}
\\g) \texttt{(define b (+ a 1))} \rightarrow \texttt{Value: 3}
\\h) \texttt{(+ a b (* a b))} \rightarrow \texttt{Value: 19}
\\i) \texttt{(= a b)} \rightarrow \texttt{Value: #f}
\\j) \texttt{(if (and (> b a) (< b (* a b)))
    b
    a)} \rightarrow \texttt{Value: 4}
\\k) \texttt{(cond ((= a 4) 6)
      ((= b 4) (+ 6 7 a))
      (else 25))} \rightarrow \texttt{Value: 16}
\\l) \texttt{(+ 2 (if (> b a) b a))} \rightarrow \texttt{Value: 6}
\\m) \texttt{(* (cond ((> a b) a)
         ((< a b) b)
         (else -1))
   (+ a 1))} \rightarrow \texttt{Value: 16}

2) Translate $\frac{5 + 4 + (2 - (3 - (6 + \frac{4}{5})))}{3 \cdot (6 - 2)(2 - 7)}$ in to prefix form.

$5+4+(2-(3-(6+\frac{4}{5})))$ \rightarrow \texttt{(+ 5 (+ 4 (- 2 (- 3 (+ 6 4/5)))))}

$3 \cdot (6 - 2)(2 - 7)$ \rightarrow \texttt{(* 3 (* (- 6 2) (- 2 7)))}

\therefore $\frac{5 + 4 + (2 - (3 - (6 + \frac{4}{5})))}{3 \cdot (6 - 2)(2 - 7)}$ \Rightarrow \texttt{(\ (+ 5 (+ 4 (- 2 (- 3 (+ 6 4/5))))) (* 3 (* (- 6 2) (- 2 7))))}

3) Define a procedure that takes three numbers as arguments and returns the sum of the squares of the two larger numbers.

\texttt{(define (sq-sum x y)
    (+ (* x x) (* y y)))

(define (gr-three a b c)
    (cond ((= a b c) (sq-sum a b))
        ((and (>= a b) (>= a c))
            (if (>= b c)
                (sq-sum a b)
                (sq-sum a c)))
        ((and (>= b c) (>= b a))
            (if (>= c a)
                (sq-sum b c)
                (sq-sum b a)))
        ((and (>= c a) (>= c b))
            (if (>= a b)
                (sq-sum c a)
                (sq-sum c b)))

    )
)}

4) Observe that our model of evaluation allows for combinations whose operators are compound expressions. Use this observation to describe the behavior of the following procedure:

\\texttt{(define (a-plus-abs-b a b)
  ((if (> b 0) + -) a b))}
\end{document}